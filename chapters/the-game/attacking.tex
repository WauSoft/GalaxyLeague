\section{Attacking} \label{attacking}
Attacking involves first rolling a Fate Die\footnote{This is also commonly called ``rolling for fate'', since a) it's a fate die, and b) it determines the fate of the attacker} to see if your attack succeeds.
The die has three different outcomes: $+$, $-$, and blank; depending on the outcome you either succeed, fail, or push respectively.
For every person helping you, you get to roll an additional attack die (see Section~\ref{sec:multidice}).

\subsection{Succeeding an Attack (Rolling $+$)}
On a successful attack, you roll against your Strength score to see if you successfully knock your opponent into the ground.
Doing so successfully, causes them to get pushed one hex and knocked over.
You then have the option of moving into the hex they just occupied.

\paragraph{Note} The effect of the roll \textit{modifies} how your roll is performed.
For example, if your character has a base Strength of 10, and you roll $++$, then you need to roll 11 or less to succeed, since in this case $Str+1=11$.

Additionally, failing your Strength check results in the attack just becoming a push (see Pushing).
Succeeding your Strength check results in the opponent to be pushed into an adjacent hex and then knocked down, leaving them out until their next turn.

\subsection{Failing an Attack (Rolling $-$)}
When you fail an attack, you must roll against Strength to see if you yourself fumble and fall on your face.

If you fail your fail roll, you will fall on your face and be knocked out for the rest of the turn.
Succeeding the fail roll means nothing happens.

\subsection{Pushing (Rolling a Blank)}
Whenever you roll a blank, your attack does not get a chance to knock someone out, you merely push them.
When you push an opponent, you move them into one of \textit{their} adjacent hexes.
After doing so, you have the option of moving into the space they just occupied.

\subsection{Rolling Multiple Dice}\label{sec:multidice} If the result of rolling multiple fate dice contains both $+$'s and $-$'s, you may choose whether it's a success ($+$) or a failure ($-$).
Then you add together the values of the remaining dice, to determine the advantage/disadvantage gained from the roll.

\example Alice chooses to use her Bruiser to attack Bob's piece with two of her pieces helping in the attack.
She rolls 3 fate dice and rolls $+, -, 0$.
She decides to take the risk, and has it be a success with a $-1$ disadvantage, which sets her Bruiser's attack to 11, 1 less than the default of 12.

\subsection{Getting Knocked Out}\label{sec:knockout}
If your character is knocked to the ground, they're out until your next turn.
A knocked out character has no tackle zone, and cannot attack.

On your next turn you have the option of having the character get back up.
Doing so counts as moving two hexes, meaning if you do not use your character after getting it back up, it counts as a wasted move.

\paragraph{Note} Because of the movement penalty for getting back up, attacking with a character that's just gotten up requires a \textit{Tackle} (see Section~\ref{energy}).