\section{Actions \& Skill Checks}\label{skill-checks}
During the game, many physical altercations will need to be modelled.
To achieve this, we use \textit{Skill Checks}; wherein you directly compare the physical characteristics---or \textit{skills}---of two or more characters.

In checking skills, you will also have to roll for your player's \fate{}.
This is done with a \textit{Fate Die}.
The outcome of this die could either add or subtracts 1 from the player's skill.

A player has three principle actions they can perform, which require skill checks: \textit{attacking}, \textit{tackling}, and \textit{dodging}.

\subsection{Harming (Fight vs. Flight)}
When a player attempts to harm another player, you compare the attacking player's \fight{} skill against the defending player's \flight{} skill, taking the relevant \fate{} rolls into account.

\paragraph{Success}
The opposing player gets knocked down, remaining in the hex where they stood.
\paragraph{Failure}
The defending coach moves the attacking player into an adjacent hex.
\paragraph{Stalemate}
Nothing happens.
The players remain in their respective hexes.

\subsection{Shove (Fight vs. Fight)}
When a player attempts to shove another player, you compare their \fight{} skills, taking the relevant \fate{} rolls into account.

\paragraph{Success} 
The opposing player gets knocked down. 
The attacking coach gets to move the downed player to an adjacent hex.
The attacking player moves into the previously occupied hex.

\begin{note}
In the event that multiple players are in a dogpile (see \secref{dogpiling}), all the participating players are elegible to move into the previously occupied hexes as the victor sees fit.
\end{note}
\paragraph{Failure}
The attacking player gets knocked down where they stand.
\paragraph{Stalemate}
Nothing happens.
The players remain in their respective hexes.

\subsection{Dodging (Flight vs. Fight)}
If your player wishes to pass through an opposing player's reach (see \secref{moving}), compare your player's \flight{} skill against their \fight{} skill, adjusted for \fate{}.

\paragraph{Success}
Your player remains unaffected and may continue moving.
\paragraph{Failure}
Your player gets knocked down where they entered the defender's reach.
\paragraph{Stalemate}
The player remains where they would have exited the defender's reach, and get to remain standing.

\subsection{Snatching (Flight vs Flight)}
\paragraph{Success}
The player that was previously holding the flag or ball is knocked down.
Your player now holds the flag or ball.
\paragraph{Failure}
Your player is knocked down.
\paragraph{Stalemate}
Both the attacking and defending players are knocked down.
The flag or ball is scattered from the player who was originally holding it.
In case of a hit, scatter again, or use the arrow on the hit icon if one is present.


\subsection{Holding the Flag}
Holding the flag increases the radius of a player's reach by one hex.
It also prevents the player from being able to initiate shoves.

In lieu of a traditional pass, your player may attempt to throw the flag at an opposing player as a ranged attack.
You roll an \textit{Attack} as normal.
If it fails, treat as a failed \throw{} instead (See \secref{flag-interaction}).

\subsection{Accounting for Multiple Players (Dogpiling)}\label{dogpiling}
In the event of mutually overlapping player reaches during an attack, the coaches may opt to have multiple players join in the dogpile.
Players are only able to join the dogpile if an enemy in their reach is already in the dogpile.
When doing this, you combine the relevant skill of all involved players and roll their combined total number of fate dice. 
See \secref{combat-step-by-step} for a breakdown.

\subsection{Combat Step-by-Step}\label{combat-step-by-step}
Here is a simplified breakdown of combat.
\begin{enumerate}
    \item Attacker declares type of attack and target
    \item Attacker declares helpers
    \item Defender declares helpers
    \item Harrier declares harrassers
    \item Combat resolves en masse
    \item Depending on outcome, the coaches may move players accordingly
    \item If successful, attacker's helpers are free to act.
\end{enumerate}

\begin{note}
Here, \textit{Harrier} refers to the coach that isn't either the attacker or defender. 
The harrier can choose to act on either side of the conflict, or both of them at once as they see fit.
\end{note}

% | Actor | Reactor | Outcome |
% |-------+---------+---------|
% | Fgt   | Fgt     | Grapple |
% | Fgt   | Flt     | Dodge   |
% | Flt   | Fgt     | Block   |