\section{Skill Checks}\label{skill-checks}
During the game, many physical altercations will need to be modelled.
To achieve this, we use \textit{Skill Checks}; wherein you directly compare the physical characteristics---or \textit{skills}---of two or more characters.

In checking skills, you will also have to roll for your player's \fate{}.
This is done with a \textit{Fate Die}.
The outcome of this die could either add or subtracts 1 from the player's skill.

A player has three principle actions they can perform, which require skill checks: \textit{attacking}, \textit{tackling}, and \textit{dodging}.

\subsection{Attacking (Fight vs. Flight)}
When a player attacks another player, you compare the attacking player's \fight{} skill against the defending player's \flight{} skill, taking the relevant \fate{} rolls into account.

\paragraph{Success}
The opposing player gets knocked down, remaining in the hex where they stood.
\paragraph{Failure}
The defending coach moves the attacking player into an adjacent hex.
\paragraph{Stalemate}
Nothing happens.
The players remain in their respective hexes.

\subsection{Tackle (Fight vs. Fight)}
When a player tackles another player, you compare their \fight{} skills, taking the relevant \fate{} rolls into account.

\paragraph{Success} 
The opposing player gets knocked down. 
The attacking coach gets to move the downed player to an adjacent hex.
The attacking player moves into the previously occupied hex.
\paragraph{Failure}
The attacking player gets knocked down where they stand.
\paragraph{Stalemate}
Nothing happens.
The players remain in their respective hexes.

\subsection{Dodging (Flight vs. Fight)}
If your player wishes to pass through an opposing player's zone\footnote{workshop this name}, compare your player's \flight{} skill against their \fight{} skill, adjusted for \fate{}.

\paragraph{Success}
Your player remains unaffected and may continue moving.
\paragraph{Failure}
Your player gets knocked down where they entered the defender's zone.
\paragraph{Stalemate}
The player remains where they would have exited the defender's zone, and get to remain standing.

\subsection{Holding the Flag}
Holding the \flag{} increases the radius of a player's zone by one hex.
It also prevents the player from being able to initiate tackles.

In lieu of a traditional pass, your player may attempt to throw the flag at an opposing player as a ranged attack.
You roll an \textit{Attack} as normal.
If it fails, treat as a failed \throw{} instead (See Section~\ref{flag-interaction}).

\subsection{Accounting for Multiple Players}
If the zones of multiple players on the same team overlap in a way where they're all subject to the same skill checks, you may add the skills together and roll their total number of \fate{} dice.

This applies to defending as well.

\begin{example}
    Two players from the \textit{Venzig Veterans} attack two players from the \textit{Kalmic Kalamities}.

    The combined \fight{} skill of the two \textit{Venzig Veterans} players is 7 ($3 + 4$), and the combined \flight{} skill of the two \textit{Kalmic Kalamities} players is 9 ($4 + 5$).
    
    The coaches the teams each get one \fate{} die from each player in the scrimmage, with the \textit{Venzig Veterans} getting 0 (\texttt{+-}), and the \textit{Kalmic Kalamities} also getting 0 (\texttt{00}).

    This was a huge risk for the \textit{Venzig Veterans} that ultimately didn't pay off, losing 7--9, unless their goal was simply to hold these players into an ongoing scrimmage.
\end{example}

% | Actor | Reactor | Outcome |
% |-------+---------+---------|
% | Fgt   | Fgt     | Grapple |
% | Fgt   | Flt     | Dodge   |
% | Flt   | Fgt     | Block   |