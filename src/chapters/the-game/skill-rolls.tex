\section{Skill Rolls}\label{skill-rolls}
Skill rolls are used to determine the outcome of certain actions, and each skill has an associated skill roll that needs to be performed in special occasions.

To perform a skill roll, a player rolls 3d6 against a character's given skill.
If the player rolls the exact value or below, it's a success; if the player rolls above it's a fail.

If a player rolls a 3 (\epsdice{1}\epsdice{1}\epsdice{1}), it is called a ``critical success''; similarly if a player rolls 18 (\epsdice{6}\epsdice{6}\epsdice{6}), it is called a ``critical failure''.
These special outcomes have special connotation in certain contexts.

\begin{example}
    If your character's Dexterity is 12 and you roll 10 with 3d6, you succeed that roll.
\end{example}

\begin{note}
    Whenever this manual refers to ``rolling against'' a certain skill, we're talking about a skill roll.
\end{note}

\subsection{Strength}
Strength rolls are required whenever you roll \texttt{+} or \texttt{-} on an attack (see \secref{attacking}).
The outcome of a strength roll determines whether you ultimately succeed or fail your roll.

\paragraph{Critical Success} If you successfully roll a 3 on an attack, you not only knock out an opponent's character, you get to take it off the field until the next third of the match.

\paragraph{Critical Failure} If you roll an 18 on an attack your own character gets knocked out until the next third.
\subsection{Dexterity}
Dexterity rolls are required whenever you try to throw, pass, catch, or intercept a flag.
A success means it goes as planned, a fail means it doesn't (see \secref{flag-interaction}).

\paragraph{Critical Success} A critical success means the throw is executed exactly the way you want it to.
You don't have to roll to catch, and the enemy doesn't get to intercept.

\subsection{Agility}
Agility rolls are required whenever you want to invoke the Sprint action (see \secref{special-action}).
A success means your character gets to move the extra distance, a fail means they fall on their face.